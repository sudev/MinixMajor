\chapter{Work Remaining}

Implement immediate files in MINIX 3 by making changes in the kernel code. An immediate file should be created by the operating system whenever a special option flag is passed to {\em creat/open } system call. The operating system should also report an error whenever the file exceeds 32 Bytes in size. A file created as an immediate file is always an immediate file and is never converted to a regular file.

\section{Inode Structure}
Below shown is the inode structure in Minix 3.

\lstinputlisting{inode.h}

The pointers to the disk block is saved in an array i\_zone each referencing a 32 bit address in the disk.This array can be used to store data when file size is lesser than 32 Bytes. 

\section{Differentiating between an immediate and regular file}

Minix file system uses the flag bits in the {\em const.h} header file for { \em i\_mode} variable of inode structure, so now we have to add flag bits to check for immediate files.

\section{Operations on an immediate file}
\subsection{Create}
Create a immediate file only when the flag bit for immediate file is set. This flag must be passed to creat/open system call.
\subsection{Read}
Instead of reading the data from the disk block the read system call should be modified to read data from the inode itself.
\subsection{Write}
The write system call should be modified to write data into the inode itself and should report an error whenever the file size goes beyond 32 Bytes.
\subsection{Delete}
When files are deleted typically indirect disk blocks need to be freed skip this step in case of immediate file and delete the inode.
