\chapter{Appendix}

\section{Message Passing in MINIX}

\subsection{Message Passing Properties}

\begin{itemize}

\item Minix uses the concept of a message to support communication between processes. This is used universally througout Minix whenever one process needs to access another.

\item    Minix user processes can only send/receive messages to system processes, not to other user processes.

\item    Minix system processes can send messages to other system processes and reply to user process messages subject to certain restrictions.

\item    The following message passing functions are implemented by the kernel:
	\begin{itemize}
	  \item   send
      \item   receive
      \item   sendrec (send/receive)
      \item   notify
      \item   echo
	\end{itemize}
       

\item The first three use the rendezvous style message passing. So sendrec in a user process will block the user if the destination system process is currently doing something other than trying to receive an new request message.

\item The \_syscall function used by user level programs actually uses the sendrec (send/receive) function to send a request and receive a reply.

\item  The notify type message does not use the rendezvous. That is, it does not get blocked if the destination is not yet ready to receive the notification.

\item The echo is used by the kernel process for passing a message from a process to itself.

\end{itemize}

\subsection{Message Format}

A message consists of: the process ID of the sending process, an identifier telling the process what sort of message it is, and then a union for holding the actual message structure. The code snippet from the file  \emph{/usr/src/include/minix/ipc.h} shows the message structure :
\begin{small}
\lstinputlisting{./codeblocks/messagestructure}
\end{small}



MINIX 3 has 10 message types (mess\_1 to mess\_10) as we can see from the message structure above. Each of these message types are unique. The code below from the same file shows the different message types :
\begin{small}
\lstinputlisting{./codeblocks/messagetypes}
\end{small}



