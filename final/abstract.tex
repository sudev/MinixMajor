\vspace{2in}
\begin{abstract}
This project aims to understand the Virtual File System of MINIX 3. The Virtual File system (VFS) is an abstraction layer over the file system implementations in the operating system. It handles all system calls related to the file system and allows for client applications to access different types of file systems in a uniform way. It also provides a common interface to several kinds of file system implementations.
\\ \linebreak
The project also gives an idea on how to implement support for immediate files in MINIX 3.  An immediate file is a file whose data is not stored in a data block, but directly inside the inode itself. With such an implementation, data fragmentation in the file system caused by small files can be solved and number of disk accesses can be reduced.




\end{abstract} 
