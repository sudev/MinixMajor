\chapter{Literature Review}

\vspace{10mm}
 Construction of a Highly Dependable Operating System, { \em Jorrit N. Herder, Herbert Bos, Ben Gras, Philip Homburg, and Andrew S. Tanenbaum - Vrije Universiteit, De Boelelaan 1081a, 1081 HV Amsterdam, The Netherlands
}\cite{chdos}
\\
\\
This paper discusses how we can design and implement a highly dependable system { \em MINIX 3} what problems are encountered, and how they are solved. They also discuss the performance effects of the changes and evaluate how the multiserver design improves operating system dependability over monolithic designs. While the emphasis is on a more reliable operating system MINIX 3 fails to better MINIX 2 in performance. One of the changes proposed to improve performance is with the filesystem where they experimented with different block sizes.

\vspace{10mm}

Immediate files, { \em  Sape. J. Mullender and Andrew S Tannenbaum - Software practice and experience April 1984 }\cite{astimme}
\\
\\
A efficient disk organization is proposed in this paper. The idea is store the first part of data inside the inode itself instead of storing the pointers there. As per the paper about 60.87 \% of the files in unix are lesser than 2048 bytes or less which means most of the files are short. Immediate blocks can be used to reduce the no of disk access considering most files being short.


How To Manipulate the Inode Structure  { \em Karthick Jayaraman, Syracuse University } \cite{inode}
\\ 
\\
The Minix filesystem is briefly explained with a note on how to manipulate and store data into the inode sturcture. Inode's internal details is explained in depth.

\vspace{10mm}

Operating Systems Design and Implementation Third Edition { \em Andrew S Tanenbaum, Albert S Woodhull }\cite{ast}
\\
\\
The operating system concepts are discussed in this book using the MINIX 3 operating system model and source code. Entire source code of MINIX 3 is available in this book. The book explains MINIX 3 filesystem in great detail. 

%\subsubsection{<Sub-subsection title>}
%even more text\footnote{<footnote here>}, and even more.

\vspace{10mm}
 Design and implementation of the MINIX Virtual File system, { \em Balazs Gerofi, August 2006 - Vrije Universiteit, The Netherlands
}\cite{vfs}
\\
\\
This paper presents a general description of the Minix Virtual File system layer.It covers the design principles, the components and their functionalities.
The various data structures used in the Virtual File System and the operations related to them are described. The communication between the VFS and the actual file system implementations are explained to a great extent.
In order to demonstrate the overall mechanism of the MINIX Virtual File system, the paper provides the main steps of the execution of a system call which gives a clear idea about the structure of the VFS.
Overall, an abstract layer has been designed and implemented, which is in charge of controlling the overall mechanism of the VFS by issuing accurate requests for the appropriate FS servers during the system calls.

\chapter{Introduction}

\section{Minix \cite{vfs} }

Different kinds of people use computers now than sev­eral decades ago, but operating systems have not fully kept pace with this change. Early users expected the computer to crash often; reboots came as naturally as waiting for the neighborhood TV re­pairman to come replace the picture tube on their home TVs. All that has changed and operating systems need to change with the times.Modern computer users are from a broad cross-section of society. Most of them have a set of mental expectations that we call The TV model.
It goes like this:
\begin{itemize}
\item You buy the device.
\item You turn it on.
\item It works perfectly for the next 10 years.
\end{itemize}

Most electronic devices fit this model well, the one exception being the personal computer. In addition to mind-numbing complexity (e.g., even networking experts have trouble configuring a wireless base station, despite the 500-page manual), they are prone to crashes and blue screens of death, issues unheard of with other electronic devices.

Most modern computer users want their systems to work all the time and never crash, ever. In engineering terms, this requires a mean time to failure (MTTF) appre­ciably longer than the expected lifetime of the computer.The average user virtually never complains that the com­puter itself is too slow (e.g., it cannot update a spread­sheet fast enough), although complaints about the speed of the Web are common. Over time, the relationship be­tween speed and reliability has reversed. Most users now consider the reliability of the computer to be far more im­portant than its speed, the reverse of 40 years ago.

Yet operating system reliability is still poor. To make the research challenge more explicit, consider a device driver that contains a fatal bug such as a store through an invalid pointer or an infinite loop. In commodity op­erating systems, when this bug is triggered, it crashes or hangs the entire system because the buggy code is run­ning in kernel mode. All user programs that were run­ning at the time the bug struck are killed, all user work is lost, and all FTP, Web, and e-mail transfers are abruptly aborted.

Studies have shown that software contains about 6-16 bugs per 1000 lines of code, it is simply infeasible to get all code to be correct. Therefore, the MINIX OS was designed in such a way that certain major faults are properly isolated, defects are detected, and failing components can be replaced on the fly, often transparent to applications and without user intervention or loss of data or work.  

\subsection{MINIX 3 and its architecture}

MINIX 3 is a microkernel based POSIX compliant operating system designed to be highly reliable, flexible, and secure. The approach is based on the ideas of modularity and fault isolation by breaking the system into many self-contained modules. In general the MINIX design is guided by the following principles:

\begin{itemize}
\item Simplicity: Keep the system as simple as possible so that it is easy to understand and thus more likely to be correct.

\item Modularity: Split the system into a collection of small, independent modules and therefore prevent failures in one module from indirectly affecting another module.

\item Least authorization: Reduce privileges of all modules as far as it is possible.

\item Fault tolerance: Design the system in a way that it withstands failures. Detect the faulty component and replace it, while the system continues running the entire time.
\end{itemize}

The operating system is structured as follows. A minimal kernel provides interrupt handlers, a mechanism for starting and stopping processes, a scheduler, and interprocess communication. Standard operating system functionality that is usually present in a monolithic kernel is moved to user space, and no longer runs at the highest privilege level. Device drivers, the file system, the network server and high-level memory management run as separate user processes that are encapsulated in their private address space.

\includegraphics[scale=0.7]{./pics/minixarch.png}

The above figure shows the structure of the operating system.\\

Although from the kernel’s point of view the server and driver processes are also just user-mode processes, logically they can be structured into three layers. The lowest level of user-mode processes are the device drivers, each one controlling some device. Drivers for IDE, floppy, and RAM disks, etc. Above the driver layer are the server processes. These include the VFS server, underlying file system implementations, process server, reincarnation server, and others. On top of the servers come the ordinary user processes including shells, compilers, utilities, and application programs.\\

Because the default mode of interprocess communication (IPC) are synchronous calls, deadlocks can occur when two or more processes simultaneously try to communicate and all processes are blocked waiting for one another. Therefore, a deadlock avoidance protocol has been carefully devised that prescribes a partial, top-down message ordering. The message ordering roughly follows the layering that is described above. Deadlock detection is also implemented in the kernel. If a process unexpectedly were to cause a deadlock, the offending is denied and an error message is returned to the caller. \\

Recovering from failures is an important reliability feature in MINIX. Servers and drivers are started and guarded by a system process called the reincarnation server. If a guarded process unexpectedly exits or crashes this is immediately detected – because the process server notifies the reincarnation server whenever a server or driver terminates – and the process is automatically restarted. Furthermore, the reincarnation server periodically polls all servers and drivers for their status. If one does not respond correctly within a specified time interval, the reincarnation server kills and restarts the misbehaving server or driver.

\section{Virtual File System in MINIX 3 \cite{vfs}}
Virtual File System is an abstraction layer – over the file system implementations – in the operating system. It provides a common interface for the applications so that they can access different types of underlying file systems in a uniform way and therefore the differences in their properties are hidden. This interface consist of the file system related system calls.\\

The VFS also provides a common interface for the underlying file systems and manages resources that are independent from the underlying file systems. This common interface ensures that new file system implementations can be added easily.

\subsection{VFS Design }

Exploiting modularity is a key idea behind MINIX, therefore the design of the Virtual File system layer is also driven by this idea. In contrast to the monolithic kernels, where the VFS layer access the implementation of the underlying file systems through function pointers, in MINIX the drivers are different processes and they communicate through IPC. During the design of the MINIX Virtual File system the most important decisions that had to be made were the followings:
\begin{itemize}
\item Which components are responsible for which functionalities.

\item Which resources are handled by the VFS and which are handled by the actual file system implementations.

\item  Where to divide the former FS process in order to get an abstract virtual layer and the actual MINIX file system implementation.
\end{itemize}

Comparing the MINIX VFS to the VFS layer in other monolithic UNIX kernels some functionalities have to be handled in a different way. In monolithic kernels the communication between the VFS layer and the underlying file system implementation is cheap, simple function calls, while sending messages between processes is more expensive. For this reason, keeping the number of messages low during a system call is important.

The MINIX Virtual File system is built in a distributed, multiserver, manner. It consists of a top-level VFS process and separate FS process for each mounted partition.

\includegraphics[scale=0.7]{./pics/vfsfs.png}

The top-level VFS process receives the requests from user programs through system calls. If actual file system operation is involved the VFS requests the corresponding FS process to do the job. This dependency is depicted in figure above. In other words all the file system calls will have to go through the virtualiztion layer first then the VFS will route it to specific file system server.

\subsection{Major steps in execution of system call}

Let's consider system call {\em stat()} with an argument {\em /usr/web/index.html} . This will help you in understanding VFS and its workflow.\\

Assume
\begin{itemize}
\item a ext2 partition mounted at /usr
\item root filesystem is Minix filesystem
\end{itemize}

\textbf{Steps}

\begin{enumerate}

\item The user process calls the stat() function of the POSIX library which builds the stat request message and sends it to the VFS process.
    (a) The VFS process copies the path name from userspace.

\item The VFS first issues a lookup for the path name. It determines that the given path is absolute, therefore the root FS process has to be requested to perform the lookup.
    (b) The root FS process copies the path name from the VFS’ address space.

\item During the lookup in the root FS process the root directory has to be read in order to find the string ”usr”. Let us assume that this information is not in the buffer cache. The root FS asks the Driver process to read the corresponding block from the disk.

\item The driver reads the block and transfers back to the FS process. It reports OK.
    (c) The driver copies the disk content into the FS’ buffer cache.

\item The root FS process examines the ”usr” directories inode data and realizes that there is a partition mounted on this directory. It sends the EENTER MOUNT message to the VFS that also contains the number of characters that were processed during the lookup.

\item The VFS looks up in the virtual mount table which FS process is responsible for the ”/usr” partition. The lookup has to be continued in that FS process ( "/usr" Mininx filesystem partition). The VFS sends the lookup request and with the rest of the path name.
    d. The ”/usr” FS process copies the path name from the VFS’ address space.

\item The FS process that handles the ”/usr” partition continues the lookup of the path name. It needs additional information from the disk, therefore it asks the driver process to read the given block and transfer it into the FS process buffer cache.

\item The driver reads the disk and transfers back to the FS process. It reports success.
    (e) The driver copies the disk content into the ”/usr” FS process’ buffer cache.

\item The "/usr" FS process finishes the lookup and transfers back the inode’s details to the VFS.

\item The VFS has all the necessary information in order to issue the actual REQ STAT request. The FS process is asked to perform the stat() operation.

\item The FS process fills in the stat buffer. Let us assume that all the information needed for this operation is in the FS process’ buffer cache, therefore no interaction is involved with the Driver process. The FS copies back to the user process’ address space. It reports success for the VFS.
    (f) The FS process copies the stat buffer to the caller process’ address space.

\item The VFS receives the response message from the FS process and sends the return value back to the POSIX library. The function reports success back to the user process.

\end{enumerate}
    
\includegraphics[scale=0.7]{./pics/vfsflow.png}

\subsection{Comparison}
Monolithic kernels are finely tuned and optimized to be efficient. Performance is one of the key issue. In contrast, the MINIX design is about reliability and security. An immediate consequence of these is that the MINIX VFS has a different structure, it has different properties. Some of these differences are given in this section.\\

As we mentioned before, kernel data structures can be easily accessed in monolithic kernels and the communication between components are simple function calls. This implies that the border between the virtual layer and the actual file system implementations is not at the same place where it is in the MINIX VFS. Monolithic kernels keep as much functionality in the VFS layer as they can.\\

Communication is free between the VFS and the underlying file system drivers therefore it makes sense to keep the virtual layer as abstract as it is possible and to reduce the functionality of the actual file system implementations. This make the implementations of a new file system easier. 

\subsection{Conclusion}

Implementation of VFS in Minix helps in creation of Virtualization layer between the kernel and other file system servers. Now implementation of any new filesystem in Minix is much easier as we don't have to deal with any other interface other than VFS and FS interface.

